\documentclass[french, 12pt, a4paper]{article}
\usepackage[T1]{fontenc}
\usepackage[utf8]{inputenc}
\usepackage{lmodern}
\renewcommand{\familydefault}{\sfdefault}
\usepackage[margin=2cm]{geometry}
\usepackage{amssymb}
\usepackage{babel}

\setlength{\parindent}{0pt}

\usepackage{minted}
\setminted[md]{
	tabsize=4,
	frame=single,
	breaklines,
}

\title{MIN1 TP2}
\author{
	Mathias Rivet\\
	Lucas Fontana
}



\begin{document}
	\maketitle



\section{Calculs à effectuer à la main}

\subsection{Conversions}

\mintinline{md}|277 = 0b100010101 = 0x115|	\\
\mintinline{md}|0x4e = 0b01001110 = 78|		\\
\mintinline{md}|0b11011001 = 217 = 0xd9|



\subsection{Addition}

\mintinline{md}|57 = 0b00111001 = 0x39|	\\
\mintinline{md}|0xa3 = 0b10100011 = 163|

Décimal
\begin{tabular}[c]{cccc}
					& 0 & 5 & 7	\\
$+_{10}$			& 1 & 6 & 3 \\
$\mathrm{C} = 0$	& 1 & 1 & 0 \\	\cline{2-4}
					& 2 & 2 & 0 \\
\end{tabular}
, hexadécimal
\begin{tabular}[c]{ccc}
					& 3 & 9	\\
$+_{16}$			& a & 3 \\
$\mathrm{C} = 0$	& 0 & 0 \\	\cline{2-3}
					& d & c \\
\end{tabular}
, binaire
\begin{tabular}[c]{ccccccccc}
					& 0 & 0 & 1 & 1	& 1 & 0 & 0 & 1	\\
$+_2$				& 1 & 0 & 1 & 0 & 0 & 0 & 1 & 1	\\
$\mathrm{C} = 0$ 	& 0 & 1 & 0 & 0 & 0 & 1 & 1 & 0	\\	\cline{2-9}
					& 1 & 1 & 0 & 1 & 1 & 1 & 0 & 0	\\
\end{tabular}

Les deux dernières retenues sont égales, $c_8 = c_7 = 0$. Les indicateurs valent $\mathrm{N} = 0$, $\mathrm{Z} = 0$ et $\mathrm{V} = 0$.

Opérande gauche: \mintinline{md}|0b00111001|		\\
Opérande droite: \mintinline{md}|0b10100011|		\\
Résultat apparent: \mintinline{md}|0b11011100|

Tous les flags $\mathrm{C}$, $\mathrm{N}$, $\mathrm{Z}$ et $\mathrm{V}$ sont à $0$ ce qui veut dire que l'opération est correcte dans $\mathbb{N}$ et $\mathbb{Z}$.

Comme tous les flags sont nuls, les réponses ne changent pas sur $9$ bits.

\paragraph{Mémo}

\begin{itemize}
	\item $\mathrm{C}$ est la dernière retenue sortante de l'addition, $\mathrm{C} = \mathrm{\bar{E}}$ (cf poly p35).
	\item $\mathrm{N}$ est égal au bit de poids fort du résultat apparent, dans les relatifs il indique si c'est un nombre négatif ou non.
	\item $\mathrm{Z} = 1$ si tout les bits du résultat apparent sont à $0$.
	\item $\mathrm{V}$ vaut $1$ si les deux dernières retenues sont de valeurs différentes, cela indique un overflow (cf poly p36).
\end{itemize}



\subsection{Soustraction}

\mintinline{md}|0xd = 0b1101 = 13|	\\
\mintinline{md}|0x8 = 0b1000 = 8|	\\
\mintinline{md}|0x7 = 0b0111 = 7|

\paragraph{0xd - 0x8}

Décimal
\begin{tabular}[c]{ccc}
					& 1 & 3	\\
$-_{10}$			& 0 & 8 \\
$\mathrm{E} = 0$	& 1 & 0 \\	\cline{2-3}
					& 0 & 5 \\
\end{tabular}
, hexadécimal
\begin{tabular}[c]{cc}
					& d	\\
$-_{16}$			& 8 \\
$\mathrm{E} = 0$	& 0 \\	\cline{2-2}
					& 5 \\
\end{tabular}
, binaire $4$ bits
\begin{tabular}[c]{ccccc}
					& 1 & 1 & 0 & 1	\\
$-_2$				& 1 & 0 & 0 & 0 \\
$\mathrm{E} = 0$ 	& 0 & 0 & 0 & 0 \\	\cline{2-5}
					& 0 & 1 & 0 & 1 \\
\end{tabular}
, binaire $5$ bits dans $\mathbb{N}$
\begin{tabular}[c]{cccccc}
					& 0 & 1 & 1 & 0 & 1	\\
$-_2$				& 0 & 1 & 0 & 0 & 0 \\
$\mathrm{E} = 0$ 	& 0 & 0 & 0 & 0 & 0 \\	\cline{2-6}
					& 0 & 0 & 1 & 0 & 1 \\
\end{tabular}
, binaire $5$ bits dans $\mathbb{Z}$
\begin{tabular}[c]{cccccc}
					& 1 & 1 & 1 & 0 & 1	\\
$-_2$				& 1 & 1 & 0 & 0 & 0 \\
$\mathrm{E} = 0$ 	& 0 & 0 & 0 & 0 & 0 \\	\cline{2-6}
					& 0 & 0 & 1 & 0 & 1 \\
\end{tabular}

Les indicateurs valent $\mathrm{N} = 0$, $\mathrm{Z} = 0$ et $\mathrm{V} = 0$.

En interprétant l'opération dans $\mathbb{N}$ ou $\mathbb{Z}$ le résultat est $5$.

La soustraction est possible dans $\mathbb{N}$ et donne un résultat correcte dans $\mathbb{Z}$.

\paragraph{0x8 - 0xd}

Décimal
\begin{tabular}[c]{ccc}
					& 0 & 8	\\
$-_{10}$			& 1 & 3 \\
$\mathrm{E} = 1$	& 0 & 0 \\	\cline{2-3}
					& 1 & 5 \\
\end{tabular}
, hexadécimal
\begin{tabular}[c]{cc}
					& 8	\\
$-_{16}$			& d \\
$\mathrm{E} = 1$	& 0 \\	\cline{2-2}
					& 5 \\
\end{tabular}
, binaire $4$ bits
\begin{tabular}[c]{ccccc}
					& 1 & 0 & 0 & 0	\\
$-_2$				& 1 & 1 & 0 & 1 \\
$\mathrm{E} = 1$ 	& 1 & 1 & 1 & 0 \\	\cline{2-5}
					& 1 & 0 & 1 & 1 \\
\end{tabular}
, binaire $5$ bits dans $\mathbb{N}$
\begin{tabular}[c]{cccccc}
					& 0 & 1 & 0 & 0 & 0	\\
$-_2$				& 0 & 1 & 1 & 0 & 1 \\
$\mathrm{E} = 1$ 	& 1 & 1 & 1 & 1 & 0 \\	\cline{2-6}
					& 1 & 1 & 0 & 1 & 1 \\
\end{tabular}
, binaire $5$ bits dans $\mathbb{Z}$
\begin{tabular}[c]{cccccc}
					& 1 & 1 & 0 & 0 & 0	\\
$-_2$				& 1 & 1 & 1 & 0 & 1 \\
$\mathrm{E} = 1$ 	& 1 & 1 & 1 & 1 & 0 \\	\cline{2-6}
					& 1 & 1 & 0 & 1 & 1 \\
\end{tabular}

Les indicateurs valent $\mathrm{N} = 1$, $\mathrm{Z} = 0$ et $\mathrm{V} = 0$.

L'opération dans $\mathbb{N}$ n'est pas interprétable alors que dans $\mathbb{Z}$ le résultat est $-5$.

La soustraction n'est pas possible dans $\mathbb{N}$ mais donne un résultat correcte dans $\mathbb{Z}$.

\paragraph{Remarque}

Le résultat apparent de la soustraction \mintinline{md}|0x8 - 0xd| en binaire commence par une infinité de $1$, il est donc interprétable sur $\mathbb{Z}$ quel que soit le nombre de bit utilisé ($n \ge 4$) pour le représenter.

\paragraph{0x7 - 0x7}

Décimal
\begin{tabular}[c]{ccc}
					& 0 & 7	\\
$-_{10}$			& 0 & 7 \\
$\mathrm{E} = 0$	& 0 & 0 \\	\cline{2-3}
					& 0 & 0 \\
\end{tabular}
, hexadécimal
\begin{tabular}[c]{cc}
					& 7	\\
$-_{16}$			& 7 \\
$\mathrm{E} = 0$	& 0 \\	\cline{2-2}
					& 0 \\
\end{tabular}
, binaire $4$ bits
\begin{tabular}[c]{ccccc}
					& 0 & 1 & 1 & 1	\\
$-_2$				& 0 & 1 & 1 & 1 \\
$\mathrm{E} = 0$ 	& 0 & 0 & 0 & 0 \\	\cline{2-5}
					& 0 & 0 & 0 & 0 \\
\end{tabular}
, binaire $5$ bits dans $\mathbb{N}$
\begin{tabular}[c]{cccccc}
					& 0 & 0 & 1 & 1 & 1	\\
$-_2$				& 0 & 0 & 1 & 1 & 1 \\
$\mathrm{E} = 0$ 	& 0 & 0 & 0 & 0 & 0 \\	\cline{2-6}
					& 0 & 0 & 0 & 0 & 0 \\
\end{tabular}
, binaire $5$ bits dans $\mathbb{Z}$
\begin{tabular}[c]{cccccc}
					& 0 & 0 & 1 & 1 & 1	\\
$-_2$				& 0 & 0 & 1 & 1 & 1 \\
$\mathrm{E} = 0$ 	& 0 & 0 & 0 & 0 & 0 \\	\cline{2-6}
					& 0 & 0 & 0 & 0 & 0 \\
\end{tabular}

Les indicateurs valent $\mathrm{N} = 0$, $\mathrm{Z} = 1$ et $\mathrm{V} = 0$.

En interprétant l'opération dans $\mathbb{N}$ ou $\mathbb{Z}$ le résultat est $0$.

La soustraction est possible dans $\mathbb{N}$ et donne un résultat correcte dans $\mathbb{Z}$.



\subsection{Soustraction par addition du complément à 2}

\paragraph{0xd + 0x7}

Sur $4$ bits le calcul est le même dans $\mathbb{N}$ et $\mathbb{Z}$.

\mintinline{md}|0xd = 0b1101|	\\
\mintinline{md}|0x7 = 0b0111|	\\
\mintinline{md}|~0x7 = 0b1000|

\begin{tabular}[c]{ccccc}
					& 1 & 1 & 0 & 1	\\
$+_2$				& 1 & 0 & 0 & 0 \\
$\mathrm{C} = 1$ 	& 0 & 0 & 1 & 1 \\	\cline{2-5}
					& 0 & 1 & 1 & 0 \\
\end{tabular}

\begin{minted}{md}
mathias@im2ag-turing-01:[~]: subc2 4 0xd 0x7
		  Bit numbers
		  3210                            if natural     if signed

		  1101 left   :  0x       d -->         13 or          -3
		+ 1000 right  :  0x       8 -->          8 or          -8
   C=1 != 0011 < c0=1   (in carries)
   V=1  ^ ---- 
 Z=0 N=0->0110  =     :  0x       6 -->          6 or          +6
\end{minted}

Sur $5$ bits dans $\mathbb{N}$.

\mintinline{md}|0xd = 0b01101|	\\
\mintinline{md}|0x7 = 0b00111|	\\
\mintinline{md}|~0x7 = 0b11000|

\begin{tabular}[c]{cccccc}
					& 0 & 1 & 1 & 0 & 1	\\
$+_2$				& 1 & 1 & 0 & 0 & 0 \\
$\mathrm{C} = 1$ 	& 1 & 0 & 0 & 1 & 1 \\	\cline{2-6}
					& 0 & 0 & 1 & 1 & 0 \\
\end{tabular}

\begin{minted}{md}
mathais@im2ag-turing-01:[~]: subc2 5 0xd 0x7
		  Bit numbers
		  4 3210                            if natural     if signed

		  0 1101 left   :  0x       d -->         13 or         +13
		+ 1 1000 right  :  0x      18 -->         24 or          -8
   C=1 == 1 0011 < c0=1   (in carries)
   V=0  ^ - ---- 
 Z=0 N=0->0 0110  =     :  0x       6 -->          6 or          +6
\end{minted}

Sur $5$ bits dans $\mathbb{Z}$.

\mintinline{md}|0xd = 0b11101|	\\
\mintinline{md}|0x7 = 0b00111|	\\
\mintinline{md}|~0x7 = 0b11000|

\begin{tabular}[c]{cccccc}
					& 1 & 1 & 1 & 0 & 1	\\
$+_2$				& 1 & 1 & 0 & 0 & 0 \\
$\mathrm{C} = 1$ 	& 1 & 0 & 0 & 1 & 1 \\	\cline{2-6}
					& 1 & 0 & 1 & 1 & 0 \\
\end{tabular}

\paragraph{Remarque}

L'extension à $5$ bits dans $\mathbb{Z}$ est à priori inutile.

\paragraph{Comparaison entre 0xd + 0x7 et 0xd - 0x8}

Dans la soustraction précédente $\mathrm{E} = 0$ alors que là $\mathrm{C} = 1$

\begin{minted}{md}
mathias@im2ag-turing-01:[~]: subc2 4 0xd 0x8
		  Bit numbers
		  3210                            if natural     if signed

		  1101 left   :  0x       d -->         13 or          -3
		+ 0111 right  :  0x       7 -->          7 or          +7
   C=1 == 1111 < c0=1   (in carries)
   V=0  ^ ---- 
 Z=0 N=0->0101  =     :  0x       5 -->          5 or          +5
\end{minted}

\paragraph{Remarque}

Ceci met en évidence le paragraphe 2.7.3 p35 du poly, $\mathrm{C} = \mathrm{\bar{E}}$.

\end{document}